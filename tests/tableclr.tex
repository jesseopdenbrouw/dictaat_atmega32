\documentclass{article}

\usepackage[table]{xcolor}
\usepackage{tabu}
\usepackage{multirow}
\colorlet{lightgrey}{lightgray}

\begin{document}

\begin{figure}[!ht]
\renewcommand\arraystretch{1.4}
\scriptsize
\centering
\begin{tabu} to 0.9\textwidth {X[,c,]X[,c,]X[,c,]X[,c,]X[,c,]X[,c,]X[,c,]X[,c,]}
7 & 6 & 5 & 4 & 3 & 2 & 1 & 0 \\
\hline
\multicolumn{1}{|c}{\cellcolor{lightgrey} JTD} & \multicolumn{1}{|c}{ISC2} & \multicolumn{1}{|c}{\cellcolor{lightgrey} $-$} & \multicolumn{1}{|c}{\cellcolor{lightgrey} JTRF} & \multicolumn{1}{|c}{\cellcolor{lightgrey} WDRF} & \multicolumn{1}{|c}{\cellcolor{lightgrey} BORF} & \multicolumn{1}{|c}{\cellcolor{lightgrey} EXTRF} & \multicolumn{1}{|c|}{\cellcolor{lightgrey} PORF} \\ \hline
R/W & R/W & R & R/W & R/W & R/W & R/W & R/W \\
0 & 0 & 0 & 0 & 0 & 0 & 0 & 0 \\
\end{tabu}
\caption{Position of the ISC2 bit in the MCUCSR register.}
\label{fig:intmcucsr}
\end{figure}
\end{document}