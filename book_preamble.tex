%%%%%%%%%%%%%%%%%%%%%%%%%%%%%%%%%%%%%%%%%%%%%%%%%%%%%%%%%%%%%%%%%%
%%%
%%%       PREAMBLE
%%%
%%%%%%%%%%%%%%%%%%%%%%%%%%%%%%%%%%%%%%%%%%%%%%%%%%%%%%%%%%%%%%%%%%

%% Suppress PTEX.FullBanner
\pdfsuppressptexinfo=1

%% 12pt charachters, A4 paper size, twoside printing, equation left aligned
%% equation indent at 1 em
\documentclass[12pt,a4paper,final,twoside,fleqn]{book}

%% Set input encoding to UTF-8
\usepackage[utf8]{inputenc}
%% Use T1 output font encoding
\usepackage[T1]{fontenc}

%%\subtitle{This is a sub title}
%%\renewcommand{\rmdefault}{crm}
\renewcommand{\author}{\bookauthor}
\renewcommand{\title}{\booktitle}
\renewcommand{\date}{\today}

%% Some defines for scaling figs
%% Scaled 1/1
\def\figscale{0.6} % Really it should be 1-exp{-1} = 0.63212055882855767840447622983854
\def\figscaleA{0.5}
\def\figscaleAA{0.43}
\def\figscaleB{0.3}
\def\figscaleC{1.0}
\def\figscaleE{0.7}
\def\figscaleF{0.6}
\def\figscaleG{0.8}

%% PDF Version and compression...
\pdfminorversion=5
\pdfobjcompresslevel=1

%% Dutch spelling of chapter, section, etc.
\usepackage[dutch]{babel}
\usepackage[english=american]{csquotes}

%% Set page layout
\usepackage[a4paper,bindingoffset=0.2in,left=1.0in,right=1.0in,top=1.0in,bottom=1.4in,
                                                               footskip=0.5in]{geometry}
                                                               
%% Use of dashed lines in tables
\usepackage{tabu}
\usepackage{tabularx}
\usepackage{booktabs}
\usepackage{multirow}

%% Use array's
\usepackage{array}

%% Define and use colors
\usepackage[x11names,table]{xcolor}

%% Include rotating (includes graphicx) packages
\usepackage{rotating}

%% Enumerate items
\usepackage[inline]{enumitem}
%% Use an asterisk before item number, see
%% http://tex.stackexchange.com/questions/61263/add-an-asterisk-left-of-an-enumerate
\setlist[enumerate]{before=\setupmodenumerate}
%
\newif\ifmoditem
\newcommand{\setupmodenumerate}{%
  \global\moditemfalse
  \let\origmakelabel\makelabel
  \def\moditem##1{\global\moditemtrue\def\mesymbol{##1}\item}%
  \def\makelabel##1{%
%    \origmakelabel{\ifmoditem\llap{\mesymbol\enspace}\fi##1}%
    \origmakelabel{\ifmoditem\llap{\mesymbol\ }\fi##1}%
    \global\moditemfalse}%
}
\newcommand{\itemstar}{\moditem{\textbf{*}}}

%% Use floats
\usepackage{float}
%% Separation between floats 12pt --> 24 pt
\setlength{\floatsep}{24.0pt plus 2.0pt minus 2.0pt}

%% Using footnotes in tables
%%%\usepackage{tablefootnote}
\usepackage{threeparttable}
%\renewcommand{\TPTminimum}{1em}
\renewcommand{\TPTnoteSettings}{\footnotesize}
\makeatletter
%\setlist[tablenotes]{label=\tnote{\alph*},ref=\alph*,itemsep=\z@,topsep=\z@skip,partopsep=\z@skip,parsep=\z@,itemindent=\z@,labelindent=\tabcolsep,labelsep=.2em,leftmargin=*,align=left,before={\footnotesize}}
\makeatother

%% Use the AMS Mathematical characters, no other AMS packages required
\usepackage{mathtools}
\setlength{\mathindent}{1em}
\DeclareMathSymbol{,}{\mathord}{letters}{"3B}

% http://archive.cs.uu.nl/mirror/CTAN/macros/latex/contrib/biblatex/doc/biblatex.pdf
\usepackage[
    backend=biber,
    backref=true,
    backrefstyle=none,
    sortcites=true,
    sorting=none,
    doi=false, % doi informatie wordt niet weergegeven
    %uniquename=true,
    %uniquelist=true,
    maxcitenames=1,
    %issn=false, werkt niet
]{biblatex}
\addbibresource{book_citations.bib}
\DefineBibliographyStrings{dutch}{
    backrefpage = {blz.},
    backrefpages = {blz.},
}
%% Do not show ISSN numbers
\AtEveryBibitem{\clearfield{issn}}
\AtEveryCitekey{\clearfield{issn}}

%% Use the Charter and Nimbus Mono fonts
\usepackage[scaled=0.92]{helvet}
\usepackage{nimbusmono}
\usepackage{charter}
\usepackage[bitstream-charter]{mathdesign}
%% Use microtype
\usepackage[stretch=10]{microtype}

%% Nice calculations with lengths
\usepackage{calc}

%%
%% Creating advanced sections with * prepended to sec number
\newcommand*{\secmark}{}
\newcommand*{\marktotoc}[1]{\renewcommand{\secmark}{#1}}
\newcommand*{\advanced}{%
  \renewcommand*{\secmark}{*}%
  \addtocontents{toc}{\protect\marktotoc{*}}%
}
\newcommand*{\basic}{%
  \renewcommand*{\secmark}{}%
  \addtocontents{toc}{\protect\marktotoc{}}%
}

% http://archive.cs.uu.nl/mirror/CTAN/macros/latex/contrib/titlesec/titlesec.pdf
\usepackage{titlesec}
\usepackage{titletoc}
\usepackage[titles]{tocloft}
\newcommand{\chapnumfont}{%     % define font for chapter number
  \usefont{T1}{pnc}{b}{n}%      % choose New Chancery, bold, normal shape
  \fontsize{100}{100}%          % font size 100pt, baselineskip 100pt
  \selectfont%                  % activate font
  \vspace*{-0.5\baselineskip}	% half baseline skip up
}
\colorlet{chapnumcol}{gray!90}  % color for chapter number
\titleformat{\chapter}[display]
{\bfseries}
{\chapnumfont\textcolor{chapnumcol}{\thechapter}}
{0ex}
{\fontfamily{phv}\selectfont\Huge\bfseries}
[{\vspace*{1ex}\titlerule[0.8pt]}]
\titleformat{\section}{\fontfamily{phv}\selectfont\large\bfseries}{\secmark\thesection}{1em}{}
\titleformat{\subsection}{\fontfamily{phv}\selectfont\bfseries}{\secmark\thesubsection}{1em}{}
\titlespacing*{\section}{0pt}{\baselineskip}{\aftersubsection}
\titlespacing*{\paragraph}{0pt}{1.0ex plus 1ex minus .2ex}{1.5em}
\newlength{\aftersubtitle}
\setlength{\aftersubtitle}{1.2\baselineskip}
\newlength{\aftersubsection}
\setlength{\aftersubsection}{\aftersubtitle}
\addtolength{\aftersubsection}{-\baselineskip}
\titlespacing*{\subsection}{0pt}{.8\baselineskip}{\aftersubsection}
\titlespacing*{\subsubsection}{0pt}{.6\baselineskip}{0pt}
%% Remove page number on first page of ToC
\AtBeginDocument{\addtocontents{toc}{\protect\thispagestyle{fancy}}}
%% Make space for page number in TOC wider
\makeatletter
\renewcommand\@pnumwidth{1cm}
\makeatother
%% Wider chapnum and secnum
\renewcommand*\cftsecnumwidth{3em}

% http://archive.cs.uu.nl/mirror/CTAN/macros/latex/contrib/footmisc/footmisc.pdf
% ruimte onder aan de pagina tussen tekst en voetnoot niet na voetnoot
% package moet VOOR fancyhdr om warning te voorkomen
\usepackage[
    bottom,
    hang,
    multiple
]{footmisc}
% inspringen
\setlength{\footnotemargin}{1em}
% ruimte tussen footnotes:
\setlength{\footnotesep}{.6\baselineskip}


%% Using fancy headers and footers
%% http://ftp.snt.utwente.nl/pub/software/tex/macros/latex/contrib/fancyhdr/fancyhdr.pdf
\usepackage{fancyhdr}
\newcommand{\bookfancy}{
    \fancyhead{} % clear all header fields
    \fancyhead[LE,RO]{\footnotesize\finalconceptdraft}
    %\fancyhead[RO,LE]{\thepage} % Right Odd, Left Even
    \fancyfoot{} % clear all footer fields
    \fancyfoot[LE,RO]{\thepage}   % page number in "outer" position of footer line
    \fancyfoot[RE,LO]{\booktitle} % other info in "inner" position of footer line
    \renewcommand{\headrulewidth}{0pt}
    \renewcommand{\footrulewidth}{0.4pt}
}
\pagestyle{fancy}
\bookfancy
\fancypagestyle{plain}{\bookfancy}
%% Clear header and footer for blank pages, possibly print "This page ..."
\makeatletter
\renewcommand*{\cleardoublepage}{\clearpage\if@twoside \ifodd\c@page\else
\hbox{}\vfill\begin{center}\thispageintentionallyleftblank\end{center}\vfill\vfill%
\thispagestyle{empty}%
\newpage%
\if@twocolumn\hbox{}\newpage\fi\fi\fi}
\makeatother

%% Indexing words...
\usepackage{imakeidx}
%% The index page is typeset...
\indexsetup{othercode=\footnotesize\thispagestyle{fancy}}
\makeindex
%\usepackage{showidx}

%% Making captions nicer...
\usepackage[font=footnotesize,format=plain,labelfont=bf,up,textfont=sl,up]{caption}
\DeclareCaptionLabelSeparator{emdash}{\ \ ---\ \ }
\usepackage[labelformat=simple,font=footnotesize,format=plain,labelfont=bf,textfont=sl]{subcaption}
\captionsetup[figure]{format=hang,justification=centering,singlelinecheck=off,skip=2ex}
\captionsetup[table]{format=hang,justification=centering,singlelinecheck=off,skip=1ex}
\captionsetup[subfigure]{format=hang,justification=centering,singlelinecheck=off,skip=2ex}
\captionsetup[subtable]{format=hang,justification=centering,singlelinecheck=off,skip=2ex}
%% Put parens around the subfig name (a) (b) etc. Needs labelformat simple
\renewcommand\thesubfigure{(\alph{subfigure})}
\renewcommand\thesubtable{(\alph{subtable})}


% Parskip et al.
\usepackage{parskip}
\makeatletter
\setlength{\parfillskip}{00\p@ \@plus 1fil}
\makeatother

%% Nice strike out of text (more used in numbers)
\usepackage{cancel}

%% Some colors
\definecolor{dkgreen}{rgb}{0,0.6,0}
\definecolor{gray}{rgb}{0.5,0.5,0.5}
\definecolor{mauve}{rgb}{0.58,0,0.82}
\definecolor{lightgrey}{rgb}{0.95,0.95,0.95}
\definecolor{GKYgray}{rgb}{0.65,0.65,0.65}
\definecolor{lightpink}{HTML}{F8E0E0}%{FBF3F3}%

%% Use computer code listings
\usepackage{listings}
%% Use textcomp for upright quotes in listings
\usepackage{textcomp}

%% Define listing style for VHDL
\lstset{%
  basicstyle=\footnotesize\ttfamily,
  numbers=left,
  numberstyle=\tiny\color{gray},
  stepnumber=1,                           
  numbersep=8pt,
  backgroundcolor=\color{DarkOliveGreen1!60!white},
  showspaces=false,
  showstringspaces=false,
  showtabs=false,
  frame=lines,
  rulecolor=\color{black},
  tabsize=4,
  captionpos=b,
  breaklines=true,
  breakatwhitespace=false,
  title=\lstname,
  upquote=true,
  %aboveskip=\baselineskip
}

%% Using hyperrefs...
\usepackage{hyperref}
\hypersetup{
	colorlinks=true,
	linkcolor=blue,
	citecolor=purple,
    pdftitle={\booktitleI{}\booktitleII},
    pdfauthor={\bookauthor},
    %pdfsubject={Beginselen van de digitale techniek},
    pdfsubject={},
    pdfkeywords={\bookkeywords},
	pdfdisplaydoctitle=true,
	pdfpagelayout=TwoPageRight
}
\urlstyle{sf}%

%% Lay things over a picture
%% http://mirrors.ctan.org/macros/latex/contrib/overpic/opic-abs.pdf
\usepackage[abs]{overpic}

%% Set \FloatBarrier to empty
\def\FloatBarrier{}

%%%
%%% LaTeX installed packages loading done here
%%%
%%% From here on, only own-written packages and commands
%%%

%% Better overline command for use in logic functions
\newcommand*{\oline}[1]{\overline{#1\mathstrut}}

%% A nice circled x for use in prime implicant tables
\newcommand{\cx}{\raisebox{-1.2pt}{\textcircled{\raisebox{1.2pt}{x}}}}

%% Prints nice "solution" header (dutch: uitwerking)
\newcommand{\uitw}[1]{\addvspace{1em}\textbf{Uitwerking opgave \ref{#1}.}\vspace*{0.2em}\newline}

%% Circled numbers
%\def\circled#1{{\ooalign{\hfil\lower.1ex\hbox{#1}\hfil\crcr\Orb}}}
\def\circled#1{\textcircled{\raisebox{.95pt}{\scriptsize #1}}}

%% Typeset propositions and postulates
\newcommand{\propositie}[1]{\par\hspace*{1em}#1\par}

%% Log notation with base
\newcommand\logbase[1]{\mathop{{}^{#1}\mathrm{log}}}

%% To input and print files.
\newcommand{\booklisting}[6][]{%
\begin{figure}[#6]%
\lstinputlisting[language=#2,caption={#3.},label=cod:#4,#1]{code/#4.#5}%
\vspace*{-\baselineskip}
\end{figure}%
}

\newcommand{\lstmodelsim}[1]{\lstinline[language=modelsim]|#1|}   %[
\newcommand{\lstvhdl}[1]{\lstinline[language=vhdl]|#1|}   %[
\newcommand{\lstc}[1]{\lstinline[language=C]|#1|}   %[
\newcommand{\lstpseu}[1]{\lstinline[language=HHSEprocPseudoLang]|#1|}   %[
\newcommand{\lstasm}[1]{\lstinline[language=HHSEprocAssembler]|#1|}   %[

%% To put single figures
%% \bookfigure{floatplace}{scale}{imagename}{caption}{label}
\newcommand{\bookfigure}[5]{%
\begin{figure}[#1]%
\centering%
\includegraphics[scale=#2]{images/#3}%
\caption{#4.}%
\label{fig:#5}%
\end{figure}%
}

%% Print hidden characters by using the white color, used for aligning numbers.
%% Please note that the dash must cast to mathbin. See:
%% http://tex.stackexchange.com/questions/21598/how-to-color-math-symbols
\newcommand\ph[1]{\if-#1\mathbin{\textcolor{white}{-}}\else\textcolor{white}{#1}\fi}

%% The askmaps package for displaying K-maps Amarican style
\usepackage{askmaps}
\askmapunitlength=0.88cm

%% Lesser and more space around the binay dot and plus and xor
\let\oldcdot\cdot
\renewcommand{\cdot}{\kern-.10em\oldcdot\kern-.10em}
\newcommand{\cdotw}{\kern.08em\oldcdot\kern.08em}
\newcommand{\cdotww}{\kern.16em\oldcdot\kern.16em}
\newcommand{\plusw}{\kern.08em+\kern.08em}
\newcommand{\plusww}{\kern.16em+\kern.16em}
\newcommand{\oplusw}{\kern.08em\oplus\kern.08em}
\newcommand{\oplusww}{\kern.16em\oplus\kern.16em}

%\usepackage{ifthen} already loaded
%% Commands to print timing parameters. They are used in dutch books.
%% Translation table:
%%   tpmax     --> tpd (combinational)
%%   tpmin     --> tpd,min (combinational)
%%   tpmax(FF) --> tclk-to-output, tco
%%   tpmin(FF) --> tclk-to-output,min, tco,min
%%   tsu       --> tsu
%%   thold     --> th (\th is already used)
%%   T         --> T
\newcommand\tpmax[1]{t_{P(max)}%
\ifthenelse{\equal{#1}{}}{}{(\mathrm{#1})}}
\newcommand\tpmin[1]{t_{P(min)}%
\ifthenelse{\equal{#1}{}}{}{(\mathrm{#1})}}
\newcommand\T[1]{T%
\ifthenelse{\equal{#1}{}}{}{_{#1}}}
\newcommand\tsu[1]{t_{su}%
\ifthenelse{\equal{#1}{}}{}{(\mathrm{#1})}}
% Note: \th already exists
\newcommand\thold[1]{t_{h}%
\ifthenelse{\equal{#1}{}}{}{(\mathrm{#1})}}
% Use \textdash to create parts like: clk-to-D
\newcommand{\textdash}{\text{-}}
%\newcommand{\txx[2]}{t_{\mathrm{#1}}(\mathrm{#2})}
%\newcommand{\txx[2]}{t_{#1}(#2)}%
\newcommand\twhmin[1]{t_{wh(min)}%
\ifthenelse{\equal{#1}{}}{}{(\mathrm{#1})}}
\newcommand\tptyp[1]{t_{P(typ)}%
\ifthenelse{\equal{#1}{}}{}{(\mathrm{#1})}}

%% Vertically align cell contents
\newcolumntype{M}[1]{>{\centering\arraybackslash$\vcenter\bgroup\hbox\bgroup}m{#1}<{\egroup\egroup$}}
\newcolumntype{C}{>{\centering\arraybackslash$\vcenter\bgroup\hbox\bgroup}c<{\egroup\egroup$}}

%% Make cells in tables vertically wider
\newcommand\Tstrut{\rule{0pt}{2.0ex}}         % = `top' strut
\newcommand\Bstrut{\rule[-0.9ex]{0pt}{0pt}}   % = `bottom' strut

%% Check if a label exists
\makeatletter
\newcommand{\iflabelexists}[3]{\@ifundefined{r@#1}{#3}{#2}}
\makeatother

%% Tikz to draw state diagrams
%\usepackage{pgf}
\usepackage{tikz}
\usetikzlibrary{positioning,shapes,arrows,automata}
\usepackage{circuitikz}
\tikzstyle{every picture}+=[font={\sffamily\small}]
\tikzstyle{every state}=[fill=none,draw=black,text=black,very thick,minimum size=1.2cm,initial distance=1cm]
% How to draw Moore states
\newif\ifmoorenodesepslash
\moorenodesepslashtrue
\ifmoorenodesepslash
\newcommand{\moorenode}[2]{$\mathsf{#1/#2}$}
\else
\newcommand{\moorenode}[2]{$\mathsf{\dfrac{#1}{#2}}$}
\fi
%% Use Venn Diagrams
\usepackage{venndiagram}

%%%%%%%%%%%%%%%%%%%%%%%%%%%%%%%%%%%%%%%%%%%%%%%%%%%%%%%%%%%%%%%%%%%%%%%%%%%
%%%%%%%%%%%%%%%%%%%%%%%%%%%%%%%%%%%%%%%%%%%%%%%%%%%%%%%%%%%%%%%%%%%%%%%%%%%
%%%%%%%%%%%%%%%%%%%%%%%%%%%%%%%%%%%%%%%%%%%%%%%%%%%%%%%%%%%%%%%%%%%%%%%%%%%
% Code voor Infobox
\makeatletter
\newsavebox{\@saveinfobox}
\newlength{\@infoboxparskip}\setlength{\@infoboxparskip}{\smallskipamount}
\newcommand{\infoboxHTMLcolor}{E7FEFF}
%% Floating minipages
%%\newfloat{floatingminipage}{!htbp}{lfm}[section]
%\restylefloat*{floatingminipage}
\floatstyle{ruled}
\newfloat{@rawinfobox}{!b}{loi}[section]
%\newfloat{@rawinfoboxh}{!h}{loi}[section]
\restylefloat*{@rawinfobox}
%\restylefloat*{@rawinfoboxh}
\floatname{@rawinfobox}{Infobox}
%\floatname{@rawinfoboxh}{Infobox}

% one length to rule them all
\newlength\hrulethickness
\setlength\hrulethickness{4\arrayrulewidth}
%\newcommand{\hlinethick}{\Xhline{\hrulethickness}}

% redefinition of the `ruled' floatstyle
% (defined in the `float' package, used by the `algorithm' package)
\renewcommand\fs@ruled{\def\@fs@cfont{\bfseries}\let\@fs@capt\floatc@ruled
  \def\@fs@pre{\hrule height \hrulethickness depth0pt }%\vskip1ex}%
  \def\@fs@post{\hrule height \hrulethickness depth0pt \relax}%
  \def\@fs@mid{\hrule height \hrulethickness depth0pt }%
  \let\@fs@iftopcapt\iftrue}

% make use of @... to set \parskip proper, affects all minipages!
%\newcommand{\@minipagerestore}{\setlength{\parskip}{\smallskipamount}}

% create a new environment
% Argument #1 = box placement, one of '!hbtp' or 'H', default '!b'
% Argument #2 = heading text, will be typeset in ...
% See http://tex.stackexchange.com/questions/11349/how-do-you-define-your-environment-such-as-to-use-for-some-parameters
\newenvironment{infobox}[2][!b]{%
% Place float at ...
\floatplacement{@rawinfobox}{#1}%
\begin{@rawinfobox}%
\begin{lrbox}{\@saveinfobox}%
% Now a dirty hack: minipage within a minipage
%\begin{minipage}[t]{\textwidth}%
%\begin{minipage}[t]{0.985\textwidth}%
\begin{minipage}[t]{\textwidth-\marginparsep-2\fboxrule-0.255pt}%
\centering%
\begin{minipage}[t]{0.95\textwidth}%
\setlength{\parskip}{\@infoboxparskip}% restore the value
\vspace*{1pt}\textsc{#2}\par
}
{%
%\vspace*{1ex}
\end{minipage}%
\end{minipage}%
\end{lrbox}%
\colorbox[HTML]{\infoboxHTMLcolor}{\usebox{\@saveinfobox}}%
\end{@rawinfobox}
}%
\makeatother
% end code for Infobox
%%%%%%%%%%%%%%%%%%%%%%%%%%%%%%%%%%%%%%%%%%%%%%%%%%%%%%%%%%%%%%%%%%%%%%%%%%%
%%%%%%%%%%%%%%%%%%%%%%%%%%%%%%%%%%%%%%%%%%%%%%%%%%%%%%%%%%%%%%%%%%%%%%%%%%%
%%%%%%%%%%%%%%%%%%%%%%%%%%%%%%%%%%%%%%%%%%%%%%%%%%%%%%%%%%%%%%%%%%%%%%%%%%%
%\usepackage{showframe}

%% Hypenation of some uncommon words in Dutch
\hyphenation{Kar-naugh-dia-gram Kar-naugh-dia-gram-men min-term min-ter-men flip-flop flip-flops im-pli-cant im-pli-can-ten twee-tal-len vier-tal-len acht-tal-len zes-tien-tal-len timing-dia-gram-men timing-dia-gram toe-stands-dia-gram toe-stands-dia-grammen}

%% The title page
\def\maketitle{%
\pagecolor{OliveDrab3}%
\null%
  \hbox{%
  \hspace*{0.05\textwidth}%
  \rule{3pt}{0.95\textheight}
  \hspace*{0.10\textwidth}%
\parbox[b]{0.70\textwidth}{%
  \vbox{%
%%%    \vspace{0.0\textheight}
    {\noindent\fontsize{70pt}{80pt}\selectfont\bfseries \booktitleI \\[0.20\baselineskip] \booktitleII}\\[3\baselineskip]
    {\fontsize{25pt}{35pt}\selectfont\itshape \booksubtitle}\\[4\baselineskip]
    {\fontsize{30pt}{35pt}\selectfont \bookauthor}\\[\baselineskip]
   \par%{\LARGE \bookinstitution}\par
   \vspace{0.3\textheight}
    {\noindent\Large \bookedition}\\[\baselineskip]
    %\vspace{0.05\textheight}
    }% end of vbox
}% end of parbox
  }% end of hbox
\thispagestyle{empty}%
\newpage
\pagecolor{white}
%%%\cleardoublepage
%%%{\centering
%%%\vfill\vspace*{5cm}
%%%{\fontsize{45pt}{55pt}\selectfont\bfseries\booktitle}\par
%%%{\fontsize{20pt}{25pt}\selectfont\itshape \booksubtitle}\par
%%%\vspace*{2cm}
%%%\Huge\bookauthor\par
%%%\Large\bookinstitution\par
%%%\vspace*{5cm}
%%%\Large\bookedition\par
%%%\vfill}
%%%\thispagestyle{empty}%
%%%\clearpage
%%%%%\pagestyle{empty}%
%%%\null%
}

%% The really backmatter
\newcommand{\bookbackmatter}{%
%%%\cleardoublepage%
%%%\hspace*{0pt}
%%%\thispagestyle{empty}
%%%\cleardoublepage
}

%%% No package loading from here
